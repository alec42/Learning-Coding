% Extra note : this preamble creates document that are meant to be used inside the multicols environment. See the documentation on internet for further information.

%% -----------------------------
%% Encoding packages
%% -----------------------------
\usepackage[utf8]{inputenc}
\usepackage[T1]{fontenc}
\usepackage{babel}
\usepackage{lmodern}
%
%%% -----------------------------
%%% Variable definition
%%% -----------------------------
%\def\auteur{Alec James van Rassel}
%\def\BackgroundColor{white}
%
%%% -----------------------------
%%% Margin and layout
%%% -----------------------------
%% Determine the margin for cheatsheet
\usepackage[landscape, hmargin=1cm, vmargin=1.7cm]{geometry}
\usepackage{multicol}

%% -----------------------------
%% URL and links
%% -----------------------------
\usepackage{hyperref}
\hypersetup{colorlinks = true, urlcolor = white, linkcolor = black}

%% -----------------------------
%% Document policy (uncomment only one)
%% -----------------------------
%	\usepackage{concrete}
%	\usepackage{mathpazo}
%	\usepackage{frcursive} %% permet d'écrire en lettres attachées
%	\usepackage{aeguill}
%	\usepackage{mathptmx}
%	\usepackage{fourier} 

%% -----------------------------
%% Math configuration
%% -----------------------------
\usepackage[fleqn]{amsmath}
\usepackage{amsthm,amssymb,latexsym,amsfonts}
\usepackage{empheq}
\usepackage{numprint}
\usepackage{dsfont} % Pour avoir le symbole du domaine Z

% Mathematics shortcuts

\newcommand{\reels}{\mathbb{R}}
\newcommand{\entiers}{\mathbb{Z}}
\newcommand{\naturels}{\mathbb{N}}
\newcommand{\eval}{\biggr \rvert}
\usepackage{cancel}
\newcommand{\derivee}[1]{\frac{\partial}{\partial #1}}
\newcommand{\prob}[1]{\Pr \left( #1 \right)}
\newcommand{\esp}[1]{\mathrm{E} \left[ #1 \right]} % espérance
\newcommand{\variance}[1]{\mathrm{Var} \left( #1   \right)}
\newcommand{\covar}[1]{\mathrm{Cov} \left( #1   \right)}
\newcommand{\laplace}{\mathcal{L}}
\newcommand{\deriv}[2][]{\frac{\partial^{#1}}{\partial #2^{#1}}}
\newcommand{\e}[1]{\mathrm{e}^{#1}}
\newcommand{\te}[1]{\text{exp}\left\{#1\right\}}
\DeclareMathSymbol{\shortminus}{\mathbin}{AMSa}{"39}

% To indicate equation number on a specific line in align environment
\newcommand\numberthis{\addtocounter{equation}{1}\tag{\theequation}}

%
% Actuarial notation packages
%
\usepackage{actuarialsymbol}
\usepackage{actuarialangle}

%
% Matrix notation for math symbols (\bm{•})
%
\usepackage{bm}
% Matrix notation variable (bold style)
\newcommand{\matr}[1]{\mathbf{#1}}


%% -----------------------------
%% tcolorbox configuration
%% -----------------------------
\usepackage[most]{tcolorbox}
\tcbuselibrary{xparse}
\tcbuselibrary{breakable}
%%	-----------------------------
%%
%%	Arguments
%%	+	breakable: allows box to be split over several pages
%%	+	segmentation style: To customize the \tcbline seperator
%%	
%%	-----------------------------
%%
%%
%% Coloured box "definition" for definitions
%%
\DeclareTColorBox{theorems}{ o}			% #1 parameter
{
	enhanced,
	title = #1,
	colback=bluebell, % color of the box
	colframe=blue(pigment),
	colbacktitle=blue!80!black,
	fonttitle = \bfseries,
	boxed title style={size=small,colframe=purple!50!black} ,
	attach boxed title to top center = {yshift=-3mm,yshifttext=-1mm},
	left=0pt,
  	right=0pt,
    box align=center,
    ams align*
%  	top=-10pt
}
\DeclareTColorBox{distributions}{ o }			% #1 parameter
{
	enhanced,
	title = #1,
	colback=ashgrey, % color of the box
%	colframe=blue(pigment),
	colframe=arsenic,	
	colbacktitle=aurometalsaurus,
	fonttitle = \bfseries,
	boxed title style={size=small,colframe=arsenic} ,
	attach boxed title to top center = {yshift=-3mm,yshifttext=-1mm},
%	left=0pt,
%  	right=0pt,
%    box align=center,
%    ams align*
%  	top=-10pt
}
\DeclareTColorBox{outcomes}{ o }			% #1 parameter
{
	enhanced,
	title = #1,
	colback=bluebell, % color of the box
%	colframe=blue(pigment),
%	colframe=asparagus,	
	colbacktitle=airforceblue,
	fonttitle = \bfseries,
	boxed title style={size=small,colframe=arsenic} ,
	attach boxed title to top center = {yshift=-3mm,yshifttext=-1mm},
%	left=0pt,
%  	right=0pt,
%    box align=center,
%    ams align*
%  	top=-10pt
}
\DeclareTColorBox{ASM_chapter}{ o }			% #1 parameter
{
	enhanced,
	title = #1,
	colback=darkseagreen, % color of the box
%	colframe=blue(pigment),
%	colframe=asparagus,	
	colbacktitle=britishracinggreen,
	fonttitle = \bfseries,
	boxed title style={size=small,colframe=arsenic} ,
	attach boxed title to top center = {yshift=-3mm,yshifttext=-1mm},
	segmentation style = {dashed, white},
	breakable
%	left=0pt,
%  	right=0pt,
%    box align=center,
%    ams align*
%  	top=-10pt
}
\DeclareTColorBox{YTB_vids}{ o }			% #1 parameter
{
	enhanced,
	title = #1,
	colback=red_rectangle, % color of the box
%	colframe=blue(pigment),
%	colframe=asparagus,	
	colbacktitle=lava,
	fonttitle = \bfseries,
	boxed title style={size=small,colframe=arsenic} ,
	attach boxed title to top center = {yshift=-3mm,yshifttext=-1mm},
	segmentation style = {dashed, white},
	breakable
%	left=0pt,
%  	right=0pt,
%    box align=center,
%    ams align*
%  	top=-10pt
}
%%
%% Coloured box "algo" for algorithms
%%
\newtcolorbox{algo}[ 1 ]
{
	colback = blue!5!white,
	colframe = blue!75!black,
	fonttitle = \bfseries,title=#1
}
%%
%% Coloured box "formula" for formulas
%%
\newtcolorbox{formula}[ 1 ]
{
	colback = green!5!white,
	colframe = darkseagreen,
	fonttitle = \bfseries,title=#1
}
%%
%% Coloured box "CHPT_SUMM" pour résumés des chapitres de l'ASM
%%
\newtcolorbox{CHPT_SUMM}[ 1 ]
{
	colback = green!5!white,
	colframe = darkseagreen,
	breakable,
	fonttitle = \bfseries,title=#1
}
%%
%% Coloured box "FORMULA_SUMM" pour résumés des formules de l'ASM
%%
\newtcolorbox{FORMULA_SUMM}[ 1 ]
{
	colback = babyblueeyes,
	colframe = airforceblue,
	breakable,
	fonttitle = \bfseries,title=#1
}
%%
%% Coloured box "YTB_SUMM" pour résumés des vidéos Youtube
%%
\newtcolorbox{YTB_SUMM}[ 1 ]
{
	colback = red!5!white,
	colframe = darkterracotta,
	breakable,
%	    frame hidden,
	fonttitle = \bfseries,title=#1
}
\newtcolorbox[auto counter, list inside = CHPT]{YTB_SUMM_AUTO_NUMB}[ 2 ][]
{
	colback = red!5!white,
	colframe = darkterracotta,
	breakable,
	enhanced,
	fonttitle = \bfseries,
	title = Video~\thetcbcounter: #2, 		
%	title = #2,
	nameref = #2,
	after upper = {\addcontentsline{toc}{subsubsection}{\thetcbcounter: #2}},
%	phantomlabel = {#2},
	#1
}
%%
%% Coloured box "FORMULA_SUMM" pour résumés des formules de l'ASM
%%
%% -----------------------------
%% Graphics and pictures
%% -----------------------------
\usepackage{graphicx}
\usepackage{pict2e}
\usepackage{tikz}
%%
%%	Creates circle 
%%	Arguments:
%%	+	size
%%	+	colour
%%	
%%	Example:
%%	+	\tikzcircle[green, fill=blue]{1.5pt}
%%	+	\tikzcircle{2pt}
%%
\newcommand{\tikzcircle}[2][red,fill=red]{\tikz[baseline=-0.5ex]\draw[#1,radius=#2] (0,0) circle ;}


%% -----------------------------
%% insert pdf pages into document
%% -----------------------------
\usepackage{pdfpages}

%% -----------------------------
%% Color configuration
%% -----------------------------
\usepackage{color, soulutf8, colortbl}

%
%	Colour definitions
%
\definecolor{ceruleanblue}{rgb}{0.16, 0.32, 0.75}
\definecolor{darkterracotta}{rgb}{0.8, 0.31, 0.36}   % red pastel ish
\definecolor{lava}{rgb}{0.81, 0.06, 0.13}
\definecolor{wildwatermelon}{rgb}{0.99, 0.42, 0.52}  % red ish
\definecolor{bostonuniversityred}{rgb}{0.8, 0.0, 0.0} % rich red
\definecolor{asparagus}{rgb}{0.53, 0.66, 0.42}		% sorta militarygreen but pastel
\definecolor{darkseagreen}{rgb}{0.56, 0.74, 0.56}    % pastel light green
\definecolor{britishracinggreen}{rgb}{0.0, 0.26, 0.15} %dark green
\definecolor{airforceblue}{rgb}{0.36, 0.54, 0.66}	% nice teal blue pastel
\definecolor{babyblueeyes}{rgb}{0.63, 0.79, 0.95}	% pastel blue-ish
\definecolor{applegreen}{rgb}{0.55, 0.71, 0.0}		% green with some aqua
\definecolor{indigo(web)}{rgb}{0.29, 0.0, 0.51}
\definecolor{cobalt}{rgb}{0.0, 0.28, 0.67}
\definecolor{azure(colorwheel)}{rgb}{0.0, 0.5, 1.0}
\definecolor{darkpastelpurple}{rgb}{0.59, 0.44, 0.84}
\definecolor{darkgreen}{rgb}{0.0, 0.2, 0.13}			
\definecolor{burntorange}{rgb}{0.8, 0.33, 0.0}		
\definecolor{burntsienna}{rgb}{0.91, 0.45, 0.32}		
\definecolor{ao(english)}{rgb}{0.0, 0.5, 0.0}		% ACT-2003
\definecolor{amber(sae/ece)}{rgb}{1.0, 0.49, 0.0} 	% ACT-2004
\definecolor{green_rectangle}{RGB}{131, 176, 84}		% ACT-2004
\definecolor{red_rectangle}{RGB}{241,112,113}		% ACT-2004
\definecolor{blue_rectangle}{RGB}{83, 84, 244}		% ACT-2004
\definecolor{blue(pigment)}{rgb}{0.2, 0.2, 0.6}
\definecolor{bluebell}{rgb}{0.64, 0.64, 0.82}
\definecolor{amethyst}{rgb}{0.6, 0.4, 0.8}
\definecolor{amethyst-light}{rgb}{0.6, 0.4, 0.8}
\definecolor{aurometalsaurus}{rgb}{0.43, 0.5, 0.5}
\definecolor{arsenic}{rgb}{0.23, 0.27, 0.29}			%	dark black-grey ish pastel
\definecolor{ashgrey}{rgb}{0.7, 0.75, 0.71}
\definecolor{beaublue}{rgb}{0.74, 0.83, 0.9}
%
% Useful shortcuts for coloured text
%
\newcommand{\orange}{\textcolor{orange}}
\newcommand{\red}{\textcolor{red}}
\newcommand{\cyan}{\textcolor{cyan}}
\newcommand{\blue}{\textcolor{blue}}
\newcommand{\green}{\textcolor{green}}
\newcommand{\purple}{\textcolor{magenta}}
\newcommand{\yellow}{\textcolor{yellow}}

%% -----------------------------
%% Enumerate environment configuration
%% -----------------------------
%
% Custum enumerate & itemize Package
%
\usepackage{enumitem}
%
% French Setup for itemize function
%
\frenchbsetup{StandardItemLabels=true}
%
% Change default label for itemize
%
\renewcommand{\labelitemi}{\faAngleRight}


%% -----------------------------
%% Tabular column type configuration
%% -----------------------------
\newcolumntype{C}{>{$}c<{$}} % math-mode version of "l" column type
\newcolumntype{L}{>{$}l<{$}} % math-mode version of "l" column type
\newcolumntype{R}{>{$}r<{$}} % math-mode version of "l" column type
\newcolumntype{f}{>{\columncolor{green!20!white}}p{1cm}}
\newcolumntype{g}{>{\columncolor{green!40!white}}m{1.2cm}}
\newcolumntype{a}{>{\columncolor{red!20!white}$}p{2cm}<{$}}	% ACT-2005
% configuration to force a line break within a single cell
\usepackage{makecell}


%% -----------------------------
%% Fontawesome for special symbols
%% -----------------------------
\usepackage{fontawesome}

%
%%% -----------------------------
%%% Footer/Header Customization
%%% -----------------------------
\usepackage{lastpage}
\usepackage{fancyhdr}
\pagestyle{fancy}
%%
%% Page background color
%%
\pagecolor{white}

%% -----------------------------
%% Section Font customization
%% -----------------------------
\usepackage{sectsty}
\sectionfont{\color{\SectionColor}}
\subsectionfont{\color{\SubSectionColor}}



%% END OF PREAMBLE
% ---------------------------------------------
% ---------------------------------------------