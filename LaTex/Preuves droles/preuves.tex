\documentclass[12pt, titlepage,french]{article}
% Extra note : this preamble creates document that are meant to be used inside the multicols environment. See the documentation on internet for further information.

%% -----------------------------
%% Encoding packages
%% -----------------------------
\usepackage[utf8]{inputenc}
\usepackage[T1]{fontenc}
\usepackage{babel}
\usepackage{lmodern}
%
%%% -----------------------------
%%% Variable definition
%%% -----------------------------
%\def\auteur{Alec James van Rassel}
%\def\BackgroundColor{white}
%
%%% -----------------------------
%%% Margin and layout
%%% -----------------------------
%% Determine the margin for cheatsheet
\usepackage[hmargin=1cm, vmargin=1.7cm]{geometry}
\usepackage{multicol}

%% -----------------------------
%% URL and links
%% -----------------------------
\usepackage{hyperref}
\hypersetup{colorlinks = true, urlcolor = white, linkcolor = black}

%% -----------------------------
%% Document policy (uncomment only one)
%% -----------------------------
%	\usepackage{concrete}
%	\usepackage{mathpazo}
%	\usepackage{frcursive} %% permet d'écrire en lettres attachées
%	\usepackage{aeguill}
%	\usepackage{mathptmx}
%	\usepackage{fourier} 

%% -----------------------------
%% Math configuration
%% -----------------------------
\usepackage[fleqn]{amsmath}
\usepackage{amsthm,amssymb,latexsym,amsfonts}
\usepackage{empheq}
\usepackage{numprint}
\usepackage{dsfont} % Pour avoir le symbole du domaine Z

% Mathematics shortcuts

\newcommand{\reels}{\mathbb{R}}
\newcommand{\entiers}{\mathbb{Z}}
\newcommand{\naturels}{\mathbb{N}}
\newcommand{\eval}{\biggr \rvert}
\usepackage{cancel}
\newcommand{\derivee}[1]{\frac{\partial}{\partial #1}}
\newcommand{\prob}[1]{\Pr \left( #1 \right)}
\newcommand{\esp}[1]{\mathrm{E} \left[ #1 \right]} % espérance
\newcommand{\variance}[1]{\mathrm{Var} \left( #1   \right)}
\newcommand{\covar}[1]{\mathrm{Cov} \left( #1   \right)}
\newcommand{\laplace}{\mathcal{L}}
\newcommand{\deriv}[2][]{\frac{\partial^{#1}}{\partial #2^{#1}}}
\newcommand{\e}[1]{\mathrm{e}^{#1}}
\newcommand{\te}[1]{\text{exp}\left\{#1\right\}}
\DeclareMathSymbol{\shortminus}{\mathbin}{AMSa}{"39}

% To indicate equation number on a specific line in align environment
\newcommand\numberthis{\addtocounter{equation}{1}\tag{\theequation}}

%
% Actuarial notation packages
%
\usepackage{actuarialsymbol}
\usepackage{actuarialangle}

%
% Matrix notation for math symbols (\bm{•})
%
\usepackage{bm}
% Matrix notation variable (bold style)
\newcommand{\matr}[1]{\mathbf{#1}}


%% -----------------------------
%% tcolorbox configuration
%% -----------------------------
\usepackage[most]{tcolorbox}
\tcbuselibrary{xparse}
\tcbuselibrary{breakable}
%%	-----------------------------
%%
%%	Arguments
%%	+	breakable: allows box to be split over several pages
%%	+	segmentation style: To customize the \tcbline seperator
%%	
%%	-----------------------------
%%
%%
%% Coloured box "definition" for definitions
%%
%%
%% Coloured box "definition2" for definitions
%%
\DeclareTColorBox{definitionNOHFILL}{ o }				% #1 parameter
{
	colframe=blue!60!green,colback=blue!5!white, % color of the box
	pad at break* = 0mm, 						% to split the box
	title = {#1},
	before title = {\faBook \quad },
	breakable
}
%%
%% Coloured box "definition2" for definitions
%%
\DeclareTColorBox{definitionNOHFILLsub}{ o }				% #1 parameter
{
	colframe=blue!40!green,colback=blue!5!white, % color of the box
	pad at break* = 0mm, 						% to split the box
	title = {#1},
	before title = {\faNavicon \quad }, %faBars  faGetPocket
	breakable
}
\DeclareTColorBox{theorems}{ o}			% #1 parameter
{
	enhanced,
	title = #1,
	colback=bluebell, % color of the box
	colframe=blue(pigment),
	colbacktitle=blue!80!black,
	fonttitle = \bfseries,
	boxed title style={size=small,colframe=purple!50!black} ,
	attach boxed title to top center = {yshift=-3mm,yshifttext=-1mm},
	left=0pt,
  	right=0pt,
    box align=center,
    ams align*
%  	top=-10pt
}
\DeclareTColorBox{distributions}{ o }			% #1 parameter
{
	enhanced,
	title = #1,
	colback=ashgrey, % color of the box
%	colframe=blue(pigment),
%	colframe=lightgray,	
	colbacktitle=aurometalsaurus,
	fonttitle = \bfseries,
	boxed title style={size=small,colframe=arsenic} ,
	attach boxed title to top center = {yshift=-3mm,yshifttext=-1mm},
%	left=0pt,
%  	right=0pt,
%    box align=center,
%    ams align*
%  	top=-10pt
}
\DeclareTColorBox{astuces}{ o }			% #1 parameter
{
	enhanced,
	title = #1,
	colback=beaublue, % color of the box
%	colframe=blue(pigment),
	colframe=ballblue,	
	colbacktitle=aurometalsaurus,
	fonttitle = \bfseries,
	boxed title style={size=small,colframe=arsenic} ,
	attach boxed title to top center = {yshift=-3mm,yshifttext=-1mm},
%	left=0pt,
%  	right=0pt,
%    box align=center,
%    ams align*
%  	top=-10pt
}
\DeclareTColorBox{outcomes}{ o }			% #1 parameter
{
	enhanced,
	title = #1,
	colback=bluebell, % color of the box
%	colframe=blue(pigment),
%	colframe=asparagus,	
	colbacktitle=airforceblue,
	fonttitle = \bfseries,
	boxed title style={size=small,colframe=arsenic} ,
	attach boxed title to top center = {yshift=-3mm,yshifttext=-1mm},
%	left=0pt,
%  	right=0pt,
%    box align=center,
%    ams align*
%  	top=-10pt
}
\DeclareTColorBox{ASM_chapter}{ o }			% #1 parameter
{
	enhanced,
	title = #1,
	colback=darkseagreen, % color of the box
%	colframe=blue(pigment),
%	colframe=asparagus,	
	colbacktitle=britishracinggreen,
	fonttitle = \bfseries,
	boxed title style={size=small,colframe=arsenic} ,
	attach boxed title to top center = {yshift=-3mm,yshifttext=-1mm},
	segmentation style = {dashed, white},
	breakable
%	left=0pt,
%  	right=0pt,
%    box align=center,
%    ams align*
%  	top=-10pt
}
\DeclareTColorBox{YTB_vids}{ o }			% #1 parameter
{
	enhanced,
	title = #1,
	colback=red_rectangle, % color of the box
%	colframe=blue(pigment),
%	colframe=asparagus,	
	colbacktitle=lava,
	fonttitle = \bfseries,
	boxed title style={size=small,colframe=arsenic} ,
	attach boxed title to top center = {yshift=-3mm,yshifttext=-1mm},
	segmentation style = {dashed, white},
	breakable
%	left=0pt,
%  	right=0pt,
%    box align=center,
%    ams align*
%  	top=-10pt
}
%%
%% Coloured box "algo" for algorithms
%%
\newtcolorbox{algo}[ 1 ]
{
	colback = blue!5!white,
	colframe = blue!75!black,
	fonttitle = \bfseries,title=#1
}
%%
%% Coloured box "formula" for formulas
%%
\newtcolorbox{formula}[ 1 ]
{
	colback = beaublue,
	colframe = airforceblue,
	fonttitle = \bfseries,title=#1
}
%%
%% Coloured box "CHPT_SUMM" pour résumés des chapitres de l'ASM
%%
\newtcolorbox{CHPT_SUMM}[ 1 ]
{
	colback = green!5!white,
	colframe = darkseagreen,
	breakable,
	fonttitle = \bfseries,title=#1
}
%%
%% Coloured box "FORMULA_SUMM" pour résumés des formules de l'ASM
%%
\newtcolorbox{FORMULA_SUMM}[ 1 ]
{
	colback = babyblueeyes,
	colframe = airforceblue,
	breakable,
	fonttitle = \bfseries,title=#1
}
%%
%% Coloured box "YTB_SUMM" pour résumés des vidéos Youtube
%%
\newtcolorbox{YTB_SUMM}[ 1 ]
{
	colback = red!5!white,
	colframe = darkterracotta,
	breakable,
%	    frame hidden,
	fonttitle = \bfseries,title=#1
}
\newtcolorbox[auto counter, list inside = CHPT]{YTB_SUMM_AUTO_NUMB}[ 2 ][]
{
	colback = red!5!white,
	colframe = darkterracotta,
	breakable,
	enhanced,
	fonttitle = \bfseries,
	title = Video~\thetcbcounter: #2, 		
%	title = #2,
	nameref = #2,
	after upper = {\addcontentsline{toc}{subsubsection}{\thetcbcounter: #2}},
%	phantomlabel = {#2},
	#1
}
%%
%% Coloured box "rappel" pour rappel de formules
%%
\DeclareTColorBox{rappel_enhanced}{ o }
{
	enhanced,
	title = #1,
	colback=lightgray, % color of the box
%	colframe=blue(pigment),
%	colframe=arsenic,	
	colbacktitle=arsenic,
	fonttitle = \bfseries,
	breakable,
	boxed title style={size=small,colframe=arsenic} ,
	attach boxed title to top center = {yshift=-3mm,yshifttext=-1mm},
}
%%
%% Coloured box "FORMULA_SUMM" pour résumés des formules de l'ASM
%%
%% -----------------------------
%% Graphics and pictures
%% -----------------------------
\usepackage{graphicx}
\usepackage{pict2e}
\usepackage{tikz}
%%
%%	Creates circle 
%%	Arguments:
%%	+	size
%%	+	colour
%%	
%%	Example:
%%	+	\tikzcircle[green, fill=blue]{1.5pt}
%%	+	\tikzcircle{2pt}
%%
\newcommand{\tikzcircle}[2][red,fill=red]{\tikz[baseline=-0.5ex]\draw[#1,radius=#2] (0,0) circle ;}


%% -----------------------------
%% insert pdf pages into document
%% -----------------------------
\usepackage{pdfpages}

%% -----------------------------
%% Color configuration
%% -----------------------------
\usepackage{color, soulutf8, colortbl}

%
%	Colour definitions
%
\definecolor{ceruleanblue}{rgb}{0.16, 0.32, 0.75}
\definecolor{darkterracotta}{rgb}{0.8, 0.31, 0.36}   % red pastel ish
\definecolor{lava}{rgb}{0.81, 0.06, 0.13}
\definecolor{wildwatermelon}{rgb}{0.99, 0.42, 0.52}  % red ish
\definecolor{bostonuniversityred}{rgb}{0.8, 0.0, 0.0} % rich red
\definecolor{asparagus}{rgb}{0.53, 0.66, 0.42}		% sorta militarygreen but pastel
\definecolor{darkseagreen}{rgb}{0.56, 0.74, 0.56}    % pastel light green
\definecolor{britishracinggreen}{rgb}{0.0, 0.26, 0.15} %dark green
\definecolor{airforceblue}{rgb}{0.36, 0.54, 0.66}	% nice teal blue pastel
\definecolor{babyblueeyes}{rgb}{0.63, 0.79, 0.95}	% pastel blue-ish
\definecolor{applegreen}{rgb}{0.55, 0.71, 0.0}		% green with some aqua
\definecolor{indigo(web)}{rgb}{0.29, 0.0, 0.51}
\definecolor{cobalt}{rgb}{0.0, 0.28, 0.67}
\definecolor{azure(colorwheel)}{rgb}{0.0, 0.5, 1.0}
\definecolor{darkpastelpurple}{rgb}{0.59, 0.44, 0.84}
\definecolor{darkgreen}{rgb}{0.0, 0.2, 0.13}			
\definecolor{burntorange}{rgb}{0.8, 0.33, 0.0}		
\definecolor{burntsienna}{rgb}{0.91, 0.45, 0.32}		
\definecolor{ao(english)}{rgb}{0.0, 0.5, 0.0}		% ACT-2003
\definecolor{amber(sae/ece)}{rgb}{1.0, 0.49, 0.0} 	% ACT-2004
\definecolor{green_rectangle}{RGB}{131, 176, 84}		% ACT-2004
\definecolor{red_rectangle}{RGB}{241,112,113}		% ACT-2004
\definecolor{blue_rectangle}{RGB}{83, 84, 244}		% ACT-2004
\definecolor{blue(pigment)}{rgb}{0.2, 0.2, 0.6}
\definecolor{bluebell}{rgb}{0.64, 0.64, 0.82}
\definecolor{amethyst}{rgb}{0.6, 0.4, 0.8}
\definecolor{amethyst-light}{rgb}{0.6, 0.4, 0.8}
\definecolor{aurometalsaurus}{rgb}{0.43, 0.5, 0.5}
\definecolor{arsenic}{rgb}{0.23, 0.27, 0.29}			%	dark black-grey ish pastel
\definecolor{ashgrey}{rgb}{0.7, 0.75, 0.71}
\definecolor{beaublue}{rgb}{0.74, 0.83, 0.9}
\definecolor{ballblue}{rgb}{0.13, 0.67, 0.8}
\definecolor{lightgray}{rgb}{0.83, 0.83, 0.83}
\definecolor{antiquefuchsia}{rgb}{0.57, 0.36, 0.51}
%
% Useful shortcuts for coloured text
%
\newcommand{\orange}{\textcolor{orange}}
\newcommand{\red}{\textcolor{red}}
\newcommand{\cyan}{\textcolor{cyan}}
\newcommand{\blue}{\textcolor{blue}}
\newcommand{\green}{\textcolor{green}}
\newcommand{\purple}{\textcolor{magenta}}
\newcommand{\yellow}{\textcolor{yellow}}

%% -----------------------------
%% Enumerate environment configuration
%% -----------------------------
%
% Custum enumerate & itemize Package
%
\usepackage{enumitem}
%
% French Setup for itemize function
%
\frenchbsetup{StandardItemLabels=true}
%
% Change default label for itemize
%
\renewcommand{\labelitemi}{\faAngleRight}


%% -----------------------------
%% Tabular column type configuration
%% -----------------------------
\newcolumntype{C}{>{$}c<{$}} % math-mode version of "l" column type
\newcolumntype{L}{>{$}l<{$}} % math-mode version of "l" column type
\newcolumntype{R}{>{$}r<{$}} % math-mode version of "l" column type
\newcolumntype{f}{>{\columncolor{green!20!white}}p{1cm}}
\newcolumntype{g}{>{\columncolor{green!40!white}}m{1.2cm}}
\newcolumntype{a}{>{\columncolor{red!20!white}$}p{2cm}<{$}}	% ACT-2005
% configuration to force a line break within a single cell
\usepackage{makecell}


%% -----------------------------
%% Fontawesome for special symbols
%% -----------------------------
\usepackage{fontawesome}

%
%%% -----------------------------
%%% Footer/Header Customization
%%% -----------------------------
\usepackage{lastpage}
\usepackage{fancyhdr}
\pagestyle{fancy}
%%
%% Page background color
%%
\pagecolor{white}

%% -----------------------------
%% Section Font customization
%% -----------------------------
\usepackage{sectsty}
%\sectionfont{\color{\SectionColor}}
%\subsectionfont{\color{\SubSectionColor}}



\usepackage{ctable}
%% END OF PREAMBLE
% ---------------------------------------------
% ---------------------------------------------
%% -----------------------------
%% Section Font customization
%% -----------------------------
\usepackage{sectsty}
\sectionfont{\color{cobalt}}
\subsectionfont{\color{indigo(web)}}
\begin{document}
\title{Ironic proofs}
\vspace{-8ex}
\date{}
\author{Alec James van Rassel}
\maketitle

\tableofcontents
\setcounter{secnumdepth}{-1}

\clearpage


%%%%	0 to an infinity symbol by tightening the belt

%\addcontentsline{toc}{section}{Unnumbered Section}
\section{Proof of homogeneity of the sinal function}

\begin{theorems}[Theorem]
	-\sin(x)&\equiv	\sin(-x) 
\end{theorems}

\begin{distributions}[Proof]
\begin{enumerate}
	\item	We define:
		\begin{align*}
		sign(x)	&=	\textcolor{blue}{+}	&
		sign(-x)	&=	\textcolor{red}{-}
		\end{align*}
	\item	We apply the \textcolor{purple}{\textbf{phonetic equivalence principle}} of $\sin$ and $sign$:
		\begin{align*}
						-\sin(x)					&=	\sin(-x)	\\		
		\therefore	-sign(x)					&=	sign(-x)		\quad	\text{\textcolor{purple}{by phonetic equivalence}}\\		
		\Rightarrow	-(\textcolor{blue}{+})	&=	\textcolor{red}{-}	\\		
		\Rightarrow	\textcolor{red}{-}		&=	\textcolor{red}{-}	
		\end{align*}		
	\item[$\therefore$]	$-\sin(x)\equiv	\sin(-x)$
\end{enumerate}
\end{distributions}
$\blacksquare$

\clearpage

\section{Proof of equivalence of addition to multiplication in the definition of the factorial}

\begin{theorems}[Theorem]
	n!	
	&=	\overset{n - 1}{\underset{i = 0}{\prod}} (n - i)	
	\equiv		\overset{n - 1}{\underset{i = 0}{\sum}} (n - i)	
\end{theorems}

\begin{distributions}[Proof]
\begin{enumerate}
	\item	It is known that:
		\begin{align*}
		n!	
		&=	\overset{n - 1}{\underset{i = 0}{\prod}} (n - i)	
		=	(n - 0) \times (n - 1) \times \dots \times 1
		\end{align*}
	\item	We apply the \textcolor{red}{\textbf{rotation property}} of multiplication
		\begin{align*}
		n!	
		=	\overset{n - 1}{\underset{i = 0}{\prod}} (n - i)	
		&=	(n - 0)\ \textcolor{black}{\times}\ (n - 1)\ \textcolor{black}{\times}\ \dots\ \textcolor{black}{\times}\ 1	\\
		&=	(n - 0)\ \textcolor{red}{\overset{\curvearrowright}{+}}\ (n - 1)\ \textcolor{red}{\overset{\curvearrowright}{+}}\ \dots\ \textcolor{red}{\overset{\curvearrowright}{+}}\ 1	\\
		&=	(n - 0)\ \textcolor{black}{+}\ (n - 1)\ \textcolor{black}{+}\ \dots\ \textcolor{black}{+}\ 1	\\
		&=	\overset{n - 1}{\underset{i = 0}{\sum}} (n - i)	
		\end{align*}		
	\item[$\therefore$]	$n!	\equiv	\overset{n - 1}{\underset{i = 0}{\sum}} (n - i)$
\end{enumerate}
\end{distributions}

\begin{formula}{Examples of application}
\begin{minipage}{0.5\linewidth}
\begin{align*}
	3!
	&=	\overset{3 - 1}{\underset{i = 0}{\prod}} (3 - i)
	=	3 \times 2 \times 1	= 6	\\
	&\equiv	\overset{3 - 1}{\underset{i = 0}{\sum}}
	=	3 + 2 + 1 = 6
\end{align*}
\end{minipage}
\begin{minipage}{0.5\linewidth}
\begin{align*}
	1!
	&=	\overset{1 - 1}{\underset{i = 0}{\prod}}
	=	1	\\
	&\equiv		\overset{1 - 1}{\underset{i = 0}{\sum}} (1 - i)	
	=	1
\end{align*}
\end{minipage}
\end{formula}
$\blacksquare$

\clearpage

\section{Proof of Product Ssubtraction}

\begin{theorems}[Theorem]
	\text{Large numbers can be substracted by substracting the product of the numbers which composes them.}
\end{theorems}

\begin{formula}{Examples of application}
\begin{align*}
	689	-	271
	&=	(6)(8)(9)	-	(2)(7)(1)	\\
	&=	432	-	14	\\
	&=	418	\\
	797	-	484
	&=	(7)(9)(7)	-	(4)(8)(4)	\\
	&=	441	-	128	\\
	&=	313	\\
	981	-	909
	&=	(9)(8)(1)	-	(9)(0)(9)	\\
	&=	72	-	0	\\
	&=	72	\\
\end{align*}
\end{formula}
$\blacksquare$

\clearpage

\section{Proof of $\sin(\pi/6)$}

\begin{theorems}[Theorem]
	\sin\left(\frac{\pi}{6}\right)
	&=	\frac{1}{2}
\end{theorems}

\begin{distributions}[Proof]
\begin{enumerate}
	\item	By the first principle of engineering, $\pi	=	3$
		\begin{align*}
		\sin\left(\frac{\pi}{6}\right)
		&=	\sin\left(\frac{3}{6}\right)	\\
		&=	\sin\left(\frac{1}{2}\right)
		\end{align*}
	\item	By the second principle of engineering, $\sin(x)	=	x$
		\begin{align*}
		\sin\left(\frac{1}{2}\right)
		&=	\frac{1}{2}
		\end{align*}
	\item[$\therefore$]	$\sin\left(\frac{\pi}{6}\right)	=	\frac{1}{2}$
\end{enumerate}
\end{distributions}
$\blacksquare$

\clearpage

\section{Proof that $5/2	=	2.5$}

\begin{theorems}[Theorem]
	5/2
	&=	2.5
\end{theorems}

\begin{distributions}[Proof]
\begin{enumerate}
	\item	We apply the rotation principle to the division sign:
		\begin{align*}
		5{\color{teal}/}2
		&=	2{\color{teal} \overset{\circlearrowright}{-}}5	\\
		&=	2	-	5
		\end{align*}
	\item	We apply the shrinkage theorem to the subtraction sign:
		\begin{align*}
		2	-	5
		&=	2	{\color{teal}\overset{\Rightarrow\Leftarrow}{\shortminus}}	5	\\
		&=	2{\color{teal}.}5
		\end{align*}
	\item[$\therefore$]	$5/2	=	2.5$
\end{enumerate}
\end{distributions}
$\blacksquare$

\clearpage

\section{Proof that $4^{2}	-	3^{2}	=	7$}

\begin{theorems}[Theorem]
	4^{2}	-	3^{2}	
	&=	7
\end{theorems}

\begin{distributions}[Proof]
\begin{enumerate}
	\item	By the new \lfbox[rappel]{additive power quotient property} of exponents \textit{(paper forthcoming)}, we rewrite:
		\begin{align*}
		4^{\color{teal}2}	-	3^{\color{teal}2}	
		&=	(4	-	3)^{\color{teal}2/2}		\\
		&=	(4	-	3)^{1}		
		\end{align*}
	\item	We then apply the \lfbox[rappel]{power-to-mathematical-sign transferral property} to the subtraction sign:
		\begin{align*}
		(4	-	3)^{\color{teal}\overset{\curvearrowleft}{1}}
		&=	(4	{\color{teal}+}	3)	\\
		&=	7
		\end{align*}
	\item[$\therefore$]	$4^{2}	-	3^{2}	=	7$
\end{enumerate}
\end{distributions}
$\blacksquare$

\clearpage

\clearpage

\section{Proof that $\frac{\shortmid}{0}	=	\infty$}

\begin{theorems}[Theorem]
	\frac{\shortmid}{0}
	&=	\infty
\end{theorems}

\begin{distributions}[Proof by reverse-order]
\begin{enumerate}
	\item	We apply the \lfbox[rappel]{rotation principle} to both sides:
		\begin{align*}
		\frac{\shortmid}{0}	\circlearrowleft
		&=	\infty	\circlearrowleft	\\
	\Rightarrow	
		\shortminus\ \vrule\: 0
		&=	8
		\end{align*}
	\item	We remove the belt from 8 on the right side to give it to 0 on the left:
		\begin{align*}
		\shortminus\ \vrule\: 0
		&=	8	\\
		\Rightarrow
		\shortminus\ \vrule\: {\color{teal}8}
		&=	{\color{teal}0}
		\end{align*}
	\item	We reapply the \lfbox[rappel]{rotation principle} to both sides:
		\begin{align*}
		\shortminus\ \vrule\: 8{\color{teal}\circlearrowright}
		&=	0	{\color{teal}\circlearrowright}\\
	\Rightarrow	
		\frac{\shortmid }{\infty}	
		&=	{\color{red}\underbrace{\color{black}0}_{\scriptsize{\shortstack{0 is invariant\\ to rotation}}}}		\\
		\end{align*}
	\item[$\therefore$]	$\frac{\shortmid}{0} =	\infty$
\end{enumerate}
\end{distributions}
$\blacksquare$

\clearpage

\section{Proof of Taylor Expansion}

\begin{distributions}[Proof]
\begin{enumerate}[label = \roman*.]
	\item	Taylor;
	\item	T\ a\ y\ l\ o\ r;
	\item	T\, a\, y\, l\, o\, r;
	\item	T\: a\: y\: l\: o\: r;
	\item	T\; a\; y\; l\; o\; r;
	\item	T\;\; a\;\; y\;\; l\;\; o\;\; r;
	\item	T\;\;\quad a\;\;\quad y\;\;\quad l\;\;\quad o\;\;\quad r;
	\item	T\;\;\quad\quad a\;\;\quad\quad y\;\;\quad\quad l\;\;\quad\quad o\;\;\quad\quad r;
	\item	T\;\;\quad\quad\quad a\;\;\quad\quad\quad y\;\;\quad\quad\quad l\;\;\quad\quad\quad o\;\;\quad\quad\quad r;
	\item	T\;\;\quad\quad\quad\quad a\;\;\quad\quad\quad\quad y\;\;\quad\quad\quad\quad l\;\;\quad\quad\quad\quad o\;\;\quad\quad\quad\quad r;
\end{enumerate}
\end{distributions}
$\blacksquare$

\clearpage

\section{Proof of the evilness of school}

\begin{theorems}[Theorem]
	\text{school}
	&=	\text{evil}
\end{theorems}

\begin{distributions}[Proof]
\begin{enumerate}
	\item	School requires time and money:
		\begin{align*}
		\text{school}
		&=	\text{time}	\times \text{money}
		\end{align*}
	\item	Time is money:
		\begin{align*}
		\text{time}
		&\equiv	\text{money}
		\end{align*}
	\item	Money is the root of all evil:
		\begin{align*}
		\text{money}
		&\equiv	\sqrt{\text{evil}}
		\end{align*}
	\item[$\therefore$]	$\text{school}	=	\text{money}^{2}	=	\text{evil}$
\end{enumerate}
\end{distributions}
$\blacksquare$

\clearpage

\end{document}