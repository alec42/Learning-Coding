\documentclass[10pt, french]{article}

\usepackage[landscape, hmargin=1cm, vmargin=1.7cm]{geometry}
\usepackage{multicol}
\usepackage{tikz}
%% -----------------------------
%% 	Colour setup for sections
%% -----------------------------
\def\SectionColor{cobalt}
\def\SubSectionColor{azure(colorwheel)}
\def\SubSubSection{azure(colorwheel)}
%% -----------------------------
%% Section Font customization
%% -----------------------------
\usepackage{sectsty}
\sectionfont{\color{\SectionColor}}
\subsectionfont{\color{\SubSectionColor}}
%% -----------------------------
%% Color definitions
%% -----------------------------
\usepackage{xcolor}
\definecolor{indigo(web)}{rgb}{0.29, 0.0, 0.51}
\definecolor{burntorange}{rgb}{0.8, 0.33, 0.0}		
\definecolor{burntsienna}{rgb}{0.91, 0.45, 0.32}		
\definecolor{cobalt}{rgb}{0.0, 0.28, 0.67}
\definecolor{azure(colorwheel)}{rgb}{0.0, 0.5, 1.0}
%% -----------------------------
%% Variable definition
%% -----------------------------
\def\auteur{Alec James van Rassel}
\def\BackgroundColor{white}
%% -----------------------------
%% 	tcolorbox configuration
%% -----------------------------
\usepackage[most]{tcolorbox}
\tcbuselibrary{xparse}
\tcbuselibrary{breakable}
%% -----------------------------
%% Color definitions
%% -----------------------------
\usepackage{xcolor}
\definecolor{indigo(web)}{rgb}{0.29, 0.0, 0.51}
%% -----------------------------
%% 	Colour setup for prestations
%%		Ajoute couleurs sur les trêmas des signes de prestations
%% -----------------------------
\usepackage{stackengine}
\newcommand\cumlaut[2][black]{\stackon[.33ex]{#2}{\textcolor{#1}{\kern-.04ex.\kern-.2ex.}}}
%
% 	Save more space than default
%
\setlength{\abovedisplayskip}{-15pt}
%% -----------------------------
%% Math configuration
%% -----------------------------
%\usepackage[fleqn]{amsmath}
\usepackage{amsthm,amssymb,latexsym,amsfonts}
\usepackage{empheq}
\usepackage{numprint}
\usepackage{dsfont} % Pour avoir le symbole du domaine Z
\usepackage{mathrsfs}
%
% Actuarial notation packages
%
\usepackage{actuarialsymbol}
\usepackage{actuarialangle}
%
% Matrix notation for math symbols (\bm{•})
%
\usepackage{bm}
% Matrix notation variable (bold style)
\newcommand{\matr}[1]{\mathbf{#1}}
\newcommand\cololine[2]{\colorlet{temp}{.}\color{#1}\bar{\color{temp}#2}\color{temp}}
\newcommand\colbar[2]{\colorlet{temp}{.}\color{#1}\bar{\color{temp}#2}\color{temp}}

\begin{document}

\begin{multicols*}{3}

\section*{Compléments de mathématiques}

\subsection*{Sommations}
\begin{align*}
\sum_{k = m}^{n} r^k &= r^{m} \frac{1 - r^{n - m + 1}}{1 - r} &
\sum_{k = 0}^{\infty}k v^k &= \frac{v}{(1 - v)^2} \\
\sum_{k = 1}^{n}k^{\textcolor{teal}{3}} &= \textcolor{teal}{\bigg(}\frac{n(n + 1)}{2}\textcolor{teal}{\bigg)^{2}} &
\sum_{k = 1}^{n}k^2 &= \frac{n(n + 1)(2n + 1)}{6} \\
\end{align*}

\section*{Mathématiques financières}

\subsection*{Intérêt simple}
\begin{align*}
	a(t)
		&=	1 + it &
	v(t)
		&=	\frac{1}{1 + it} \\
	\text{Prix}
		&=	100 \left( 1 - \frac{it}{365} \right)^{-1}
\end{align*}

\subsection*{facteurs d'actualisation et d'accumulation}
% [t] tells LaTex to place minipage at the top of the page
\begin{minipage}[t]{.5\linewidth}
\begin{align*}
	a(t) 
		&= 	(1 + i)^{t} 		\\
		&= 	(1 - d)^{-t}	\\
		&= 	e^{\int_{0}^{t} \delta_s ds} 
\end{align*}
\end{minipage}
\begin{minipage}[t]{.5\linewidth}
	\begin{align*}
	v(t) 
		&=	(1 + i)^{-t}		\\
		&=	(1 - d)^{t}	 	\\
		&= 	e^{-\int_{0}^{t} \delta_s ds} 
\end{align*}
\end{minipage}

\subsection*{Conversion de taux}
\begin{align*}
%	i 	&= \left( 1 + \frac{i^{(m)}}{m} \right)^{m} - 1 &
	d	&= 	\frac{i}{1 + i} &
	i^{\mathrm{R}}
		&=	\frac{i - r}{1 + r}
\end{align*}

%\subsection*{Conversion de taux}
\begin{align*}
&\text{Taux d'intérêt \underline{effectif} \textit{annuel}} & & i = \left( 1 + \frac{i^{(m)}}{m} \right)^{m} - 1 \\
&\text{Taux d'intérêt \underline{nominal} \textit{annuel}} & & i^{(m)} = m \left( (1 + i)^{1/m} - 1 \right)
\end{align*}

\subsection*{Rentes constantes}

\begin{align*}
	\cumlaut[cyan]{a}_{\angl{n}}^{\textcolor{red}{(m)}} 
		&= \frac{1 - v^n}{ (i^{\textcolor{red}{(m)}}{\color{black}|}{\color{cyan}d}^{\textcolor{red}{(m)}}{\color{black})}}	&
	\cumlaut[cyan]{s}_{\angln}^{\textcolor{red}{(m)}} 
		&=	\frac{(1+i)^{n} - 1}{(i^{\textcolor{red}{(m)}} | \textcolor{cyan}{d}^{\textcolor{red}{(m)}})}	\\
	\cumlaut[cyan]{a}_{\angl{\infty}} 
		&= \frac{1}{(i|\textcolor{cyan}{d})}
\end{align*}

\subsection*{Rentes continues}
\begin{align*}
	(\colbar{indigo(web)}{I}\colbar{indigo(web)}{s})_{\angl{n}i} &= \frac{\colbar{indigo(web)}{s}_{\angl{n}i} - n}{\textcolor{indigo(web)}{\delta}} &
	(\colbar{indigo(web)}{D}\colbar{indigo(web)}{s})_{\angl{n}i} &= \frac{n v^{n} - \colbar{indigo(web)}{s}_{\angl{n}i}}{\textcolor{indigo(web)}{\delta}} \\
	(\colbar{indigo(web)}{I}\colbar{indigo(web)}{a})_{\angl{n}i} &= \frac{\colbar{indigo(web)}{a}_{\angl{n}i} - n v^{n}}{\textcolor{indigo(web)}{\delta}} &
	(\colbar{indigo(web)}{D}\colbar{indigo(web)}{a})_{\angl{n}i} &= \frac{n - \colbar{indigo(web)}{a}_{\angl{n}i}}{\textcolor{indigo(web)}{\delta}} 
\end{align*}	

\subsection*{Rentes (dé)croissantes annuellement}
\begin{align*}
	(I^{\textcolor{blue}{(m)}}\cumlaut[cyan]{a})_{\angl{n}}^{\textcolor{red}{(m)}} 
		&= \frac{\cumlaut[black]{a}_{\angl{n}}^{\textcolor{blue}{(m)}} - nv^n}{(i | \textcolor{cyan}{d}^{\textcolor{red}{(m)}})} &
	(D^{\textcolor{blue}{(m)}}\cumlaut[cyan]{a})_{\angl{n}}^{\textcolor{red}{(m)}} 
		&= \frac{n - a_{\angl{n}}^{\textcolor{blue}{(m)}}}{(i | \textcolor{cyan}{d}^{\textcolor{red}{(m)}})}
\end{align*}

%Rentes accumulées croissantes arithmétiquements
\begin{align*}
	(I^{\textcolor{blue}{(m)}}\cumlaut[cyan]{s})_{\angl{n}}^{\textcolor{red}{(m)}} 
		&= \frac{\cumlaut[black]{s}_{\angl{n}}^{\textcolor{blue}{(m)}} - n}{(i | \textcolor{cyan}{d}^{\textcolor{red}{(m)}})} &
	(D^{\textcolor{blue}{(m)}}\cumlaut[cyan]{s})_{\angl{n}}^{\textcolor{red}{(m)}} 
		&= \frac{n(1+i)^{n} - s_{\angl{n}}^{\textcolor{blue}{(m)}}}{(i | \textcolor{cyan}{d}^{\textcolor{red}{(m)}})} 
\end{align*}

\subsection*{Rentes croissantes continûment}
\begin{align*}
	(I\cumlaut[cyan]{a})_{\angl{\infty}} 
		&= \frac{1}{d(\textcolor{black}{i}|\textcolor{cyan}{d})}
\end{align*}

Paiement en continu, valeurs accumulée et actualisée
\begin{align*}
		(\colbar{indigo(web)}{I}\colbar{indigo(web)}{s})_{\angl{n}\delta_{s}, h(t)} &= \int_{0}^{n} h(t) \mathrm{e}^{\int_{t}^{n}\delta_{s}ds}dt 	\\
	(\colbar{indigo(web)}{I}\colbar{indigo(web)}{a})_{\angl{n}\delta_{s}, h(t)} &= \int_{0}^{n} h(t) \mathrm{e}^{-\int_{0}^{t} \delta_{s}ds}dt	
\end{align*}

\subsection*{Rentes avec croissance géométrique}
\begin{align*}
	\cumlaut[cyan]{a}_{\angl{n}i^{\mathrm{R}}}
%		&=	\frac{1 - (1 + i^{\mathrm{R}})^{-n}}{\left( \frac{i^{\mathrm{R}}}{1 + i^{\mathrm{R}}} \right)} {\color{cyan}\frac{1}{1 + r}	}	
		&=	\frac{1 - \left[\frac{1 + r}{1 + i}\right]^{n}}{i - r} \textcolor{cyan}{(1 + i)}
		&
	\cumlaut[cyan]{s}_{\angln i^{\mathrm{R}}}
%		&=	\frac{(1+i)^{n} - (1 + r)^{n}}{i - r}
		&=	\frac{(1+i)^{n} - (1 + r)^{n}}{i - r} \textcolor{cyan}{(1 + i)}
\end{align*}

\subsection*{T-Bills}

\begin{align*}
	\textrm{Prix}	
	&=	100 \left( 1 - \frac{dt}{360} \right)^{t}
\end{align*}

\subsection*{Obligations}

Formule de base

\begin{enumerate}
	\item[\text{P}] Prix de l'obligation 
	\item[\text{F}] Valeur nominale de l'obligation \textit{(face value)}
	\item[$r$] Taux de coupon par période de paiement \textit{(coupon rate)}
	\item[$i$] Taux d'intérêt par période de paiement \textit{(interest rate)}
	\begin{enumerate}
		\item[$Fr$] Montant par paiement.
	\end{enumerate}
	\item[\text{C}] Valeur de remboursement de l'obligation \textit{(redemption value)}
\end{enumerate}

\begin{align*}
	\text{P}
		&=	\text{F} r \ax{\angln i} + \text{C} v^{n} \\
		&=	\text{C} + (\text{F} r - \text{C} i) \ax{\angln i} +  v^{n} 
\end{align*}

\subsubsection*{Amortissement d'obligations}

Book value
\begin{align*}
	\text{BV}_{t}
		&=	(\text{F}r - \text{C}) \ax[n - t]{j} + \text{C}
\end{align*}

\end{multicols*}

\end{document}
