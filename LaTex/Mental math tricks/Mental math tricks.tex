\documentclass[10pt, french]{article}
%% -----------------------------
%% Préambule
%% -----------------------------
% Extra note : this preamble creates document that are meant to be used inside the multicols environment. See the documentation on internet for further information.

%% -----------------------------
%% Encoding packages
%% -----------------------------
\usepackage[utf8]{inputenc}
\usepackage[T1]{fontenc}
\usepackage{babel}
\usepackage{lmodern}
%
%%% -----------------------------
%%% Variable definition
%%% -----------------------------
%\def\auteur{Alec James van Rassel}
%\def\BackgroundColor{white}
%
%%% -----------------------------
%%% Margin and layout
%%% -----------------------------
%% Determine the margin for cheatsheet
%\usepackage[landscape, hmargin=1cm, vmargin=1.7cm]{geometry}
%\usepackage{multicol}

%% -----------------------------
%% URL and links
%% -----------------------------
\usepackage{hyperref}
\hypersetup{colorlinks = true, urlcolor = gray!70!white, linkcolor = black}

%% -----------------------------
%% Document policy (uncomment only one)
%% -----------------------------
%	\usepackage{concrete}
%	\usepackage{mathpazo}
%	\usepackage{frcursive} %% permet d'écrire en lettres attachées
%	\usepackage{aeguill}
%	\usepackage{mathptmx}
%	\usepackage{fourier} 

%% -----------------------------
%% Math configuration
%% -----------------------------
\usepackage[fleqn]{amsmath}
\usepackage{amsthm,amssymb,latexsym,amsfonts}
\usepackage{empheq}
\usepackage{numprint}
\usepackage{dsfont} % Pour avoir le symbole du domaine Z

% Mathematics shortcuts

\newcommand{\reels}{\mathbb{R}}
\newcommand{\entiers}{\mathbb{Z}}
\newcommand{\naturels}{\mathbb{N}}
\newcommand{\eval}{\biggr \rvert}
\usepackage{cancel}
\newcommand{\derivee}[1]{\frac{\partial}{\partial #1}}
\newcommand{\prob}[1]{\Pr \left( #1 \right)}
\newcommand{\esp}[1]{\mathrm{E} \left[ #1 \right]} % espérance
\newcommand{\variance}[1]{\mathrm{Var} \left( #1   \right)}
\newcommand{\covar}[1]{\mathrm{Cov} \left( #1   \right)}
\newcommand{\laplace}{\mathcal{L}}
\newcommand{\deriv}[2][]{\frac{\partial^{#1}}{\partial #2^{#1}}}
\newcommand{\e}[1]{\mathrm{e}^{#1}}
\newcommand{\te}[1]{\text{exp}\left\{#1\right\}}
\DeclareMathSymbol{\shortminus}{\mathbin}{AMSa}{"39}

% To indicate equation number on a specific line in align environment
\newcommand\numberthis{\addtocounter{equation}{1}\tag{\theequation}}

%
% Actuarial notation packages
%
\usepackage{actuarialsymbol}
\usepackage{actuarialangle}

%
% Matrix notation for math symbols (\bm{•})
%
\usepackage{bm}
% Matrix notation variable (bold style)
\newcommand{\matr}[1]{\mathbf{#1}}


%% -----------------------------
%% tcolorbox configuration
%% -----------------------------
\usepackage[most]{tcolorbox}
\tcbuselibrary{xparse}
\tcbuselibrary{breakable}

%%
%% Coloured box "definition" for definitions
%%
\DeclareTColorBox{theorems}{ o}			% #1 parameter
{
	enhanced,
	title = #1,
	colback=bluebell, % color of the box
	colframe=blue(pigment),
	colbacktitle=blue!80!black,
	fonttitle = \bfseries,
	boxed title style={size=small,colframe=purple!50!black} ,
	attach boxed title to top center = {yshift=-3mm,yshifttext=-1mm},
	left=0pt,
  	right=0pt,
    box align=center,
    ams align*
%  	top=-10pt
}
\DeclareTColorBox{distributions}{ o }			% #1 parameter
{
	enhanced,
	title = #1,
	colback=ashgrey, % color of the box
%	colframe=blue(pigment),
	colframe=arsenic,	
	colbacktitle=aurometalsaurus,
	fonttitle = \bfseries,
	boxed title style={size=small,colframe=arsenic} ,
	attach boxed title to top center = {yshift=-3mm,yshifttext=-1mm},
%	left=0pt,
%  	right=0pt,
%    box align=center,
%    ams align*
%  	top=-10pt
}
%%
%% Coloured box "algo" for algorithms
%%
\newtcolorbox{algo}[ 1 ]
{
	colback = blue!5!white,
	colframe = blue!75!black,
	fonttitle = \bfseries,title=#1
}
%%
%% Coloured box "formula" for formulas
%%
\newtcolorbox{formula}[ 1 ]
{
	colback = green!5!white,
	colframe = green!70!black,
	fonttitle = \bfseries,title=#1
}

%% -----------------------------
%% Graphics and pictures
%% -----------------------------
\usepackage{graphicx}
\usepackage{pict2e}
\usepackage{tikz}

%% -----------------------------
%% insert pdf pages into document
%% -----------------------------
\usepackage{pdfpages}

%% -----------------------------
%% Color configuration
%% -----------------------------
\usepackage{color, soulutf8, colortbl}

%
%	Colour definitions
%
\definecolor{indigo(web)}{rgb}{0.29, 0.0, 0.51}
\definecolor{cobalt}{rgb}{0.0, 0.28, 0.67}
\definecolor{azure(colorwheel)}{rgb}{0.0, 0.5, 1.0}
\definecolor{darkpastelpurple}{rgb}{0.59, 0.44, 0.84}
\definecolor{darkgreen}{rgb}{0.0, 0.2, 0.13}			
\definecolor{burntorange}{rgb}{0.8, 0.33, 0.0}		
\definecolor{burntsienna}{rgb}{0.91, 0.45, 0.32}		
\definecolor{ao(english)}{rgb}{0.0, 0.5, 0.0}		% ACT-2003
\definecolor{amber(sae/ece)}{rgb}{1.0, 0.49, 0.0} 	% ACT-2004
\definecolor{green_rectangle}{RGB}{131, 176, 84}		% ACT-2004
\definecolor{red_rectangle}{RGB}{241,112,113}		% ACT-2004
\definecolor{blue_rectangle}{RGB}{83, 84, 244}		% ACT-2004
\definecolor{blue(pigment)}{rgb}{0.2, 0.2, 0.6}
\definecolor{bluebell}{rgb}{0.64, 0.64, 0.82}
\definecolor{amethyst}{rgb}{0.6, 0.4, 0.8}
\definecolor{amethyst-light}{rgb}{0.6, 0.4, 0.8}
\definecolor{aurometalsaurus}{rgb}{0.43, 0.5, 0.5}
\definecolor{arsenic}{rgb}{0.23, 0.27, 0.29}			%	dark black-grey ish pastel
\definecolor{ashgrey}{rgb}{0.7, 0.75, 0.71}
%
% Useful shortcuts for coloured text
%
\newcommand{\orange}{\textcolor{orange}}
\newcommand{\red}{\textcolor{red}}
\newcommand{\cyan}{\textcolor{cyan}}
\newcommand{\blue}{\textcolor{blue}}
\newcommand{\green}{\textcolor{green}}
\newcommand{\purple}{\textcolor{magenta}}
\newcommand{\yellow}{\textcolor{yellow}}

%% -----------------------------
%% Enumerate environment configuration
%% -----------------------------
%
% Custum enumerate & itemize Package
%
\usepackage{enumitem}
%
% French Setup for itemize function
%
\frenchbsetup{StandardItemLabels=true}
%
% Change default label for itemize
%
\renewcommand{\labelitemi}{\faAngleRight}


%% -----------------------------
%% Tabular column type configuration
%% -----------------------------
\newcolumntype{C}{>{$}c<{$}} % math-mode version of "l" column type
\newcolumntype{L}{>{$}l<{$}} % math-mode version of "l" column type
\newcolumntype{R}{>{$}r<{$}} % math-mode version of "l" column type
\newcolumntype{f}{>{\columncolor{green!20!white}}p{1cm}}
\newcolumntype{g}{>{\columncolor{green!40!white}}m{1.2cm}}
\newcolumntype{a}{>{\columncolor{red!20!white}$}p{2cm}<{$}}	% ACT-2005
% configuration to force a line break within a single cell
\usepackage{makecell}


%% -----------------------------
%% Fontawesome for special symbols
%% -----------------------------
\usepackage{fontawesome}

%
%%% -----------------------------
%%% Footer/Header Customization
%%% -----------------------------
%\usepackage{lastpage}
%\usepackage{fancyhdr}
%\pagestyle{fancy}
%%
%% Page background color
%%
%\pagecolor{\BackgroundColor}




%% END OF PREAMBLE
% ---------------------------------------------
% ---------------------------------------------
%% -----------------------------
%% Redefine from template
%% -----------------------------
\def\auteur{Alec James van Rassel}
%% -----------------------------
%% Variable definition
%% -----------------------------
\def\cours{Secrets of mental math}
%\def\sigle{Arthur Ben}
\usepackage[landscape, hmargin=1cm, vmargin=1.7cm]{geometry}
\usepackage{multicol}

%% -----------------------------
%% Colour setup for sections
%% -----------------------------
\def\SectionColor{burntorange}
\def\SubSectionColor{burntsienna}
\def\SubSubSection{burntsienna}

% 
% Débute numérotation des chapitres à 2 pour suivre les notes de Marie-Piere.
% 
\setcounter{section}{1}

%% -----------------------------
%% Début du document
%% -----------------------------
\begin{document}

\begin{multicols*}{3} 
\section*{Quick tricks}
\begin{CHPT_SUMM}{Multiplication by 11}
We insert the sum between the 2 original numbers.
For example:
	\begin{align*}
		33 \times 11	
		&=	3 \textcolor{cyan}{6}3
	\end{align*}
If the sum is more than 9, we simply carry over. 
For example:
	\begin{align*}
		93 \times 11	
		&=	\textcolor{teal}{10} \textcolor{cyan}{2} 3
	\end{align*}
where we had $9 + 3 = 12$.
If we want to multiply larger numbers, the same rule carries over.
For example:
	\begin{align*}
		427 \times 11
		&=	4 \textcolor{teal}{6} \textcolor{cyan}{9}7
	\end{align*}
where we had $\textcolor{teal}{4 + 2 = 6}$ and $\textcolor{cyan}{2 + 7 = 9}$.
\end{CHPT_SUMM}

\begin{CHPT_SUMM}{Multiplying 2 numbers whose first digit is the same and whose second sums to 10}
The procedure is to multiply the first digit by the next number and to multiply the second digits together.
For example :
	\begin{align*}
	35 \times 35
	&=	\textcolor{teal}{12}\textcolor{cyan}{25}
	\end{align*}
with $\textcolor{teal}{3 \times 4 = 12}$ and $\textcolor{teal}{5 \times 5 = 25}$.

It is important to have 2 digits for the multiplication of the second digit. 
For example:
	\begin{align*}
	31 \times 39
	&=	\textcolor{teal}{12}\textcolor{cyan}{09}
	\end{align*}
with $\textcolor{teal}{3 \times 4 = 12}$ and $\textcolor{teal}{1 \times 9 = 09}$.
\end{CHPT_SUMM}

\begin{CHPT_SUMM}{Squaring any number}
The algebraic formula is, where $A$ and $d$ are integers, $A^{2} = (A + d) \times (A - d) + d^{2}$.
For example:
	\begin{align*}
	98^{2}	
	&=	100 \times 96 + 2^{2}
	=	960\textcolor{teal}{4}
	\end{align*}
with the number $\textcolor{teal}{2}$ being added and substracted and from $98$.

We can generalize this to 3 digit numbers.
For example:
	\begin{align*}
	212^{2}	
	&=	2\textcolor{teal}{00} \times 224 + 12^{2}	\\
	&=	448\textcolor{teal}{00} + 144
	=	44944
	\end{align*}
	
Because we round down to the nearest hundred, we only have to multiply the first digit.
\end{CHPT_SUMM}

\begin{CHPT_SUMM}{Multiplying numbers between 10 and 20}
We add the second digit of the second number to the first and add the multiple of the second digit for both numbers.
For example:
	\begin{align*}
	17 \times 14
	&=	\textcolor{cyan}{(17 + 4)} \times \textcolor{amethyst}{10} + \textcolor{teal}{7 \times 4}	\\
	&=	\textcolor{cyan}{21}\textcolor{amethyst}{0} + \textcolor{teal}{28}	
	=	238
	\end{align*}
\end{CHPT_SUMM}


\begin{CHPT_SUMM}{Almost perfect squares}
Recall that \icbox{$x^{2}	-	1	=	(x	-	1)(x		+	1)$}. In fact, $x^{2}$ can be visualized as a square from which we substract one to form a rectangle of lengths $x	-	1$ and $x	+	1$.
\end{CHPT_SUMM}

\pagebreak
\section{Dates and units}
Since 1582 (introduction of the Gregorian calendar).\\

A year is a leap year if :
\begin{itemize}
	\item	The year can be divided by 4 (e.g., 2016, 2020, 2024, etc.).
	\item	The year cannot be divided by 100 (e.g., 2100, 2200, etc.).
	\item	Except if it can be divided by 400 (e.g., 2000, 2400, etc.).	
\end{itemize}

\subsection{Find day of the week of any date}
We pose : 
\begin{description}
	\item[$h$]	Day of the week with $\{0 = \text{Sat.}, 1 = \text{Sun.}, 2 = \text{Mon.}, \dots, 6 = \text{Fri.}\}$.
	\item[$d_{m}$]	Day of the month.
	\item[$m$]	Month with $\{3 = \text{Mar.}, 4 = \text{Apr.}, 2 = \text{Mon.}, \dots, 14 = \text{Feb.}\}$.
		\begin{itemize}
		\item	So we'd consider February to be the 14th month of the previous year.
		\end{itemize}
	\item[$K$]	Year of the century, $Y mod 100$.
	\item[$J$]	Zero-based century, $\lfloor Y / 100\rfloor$.
\end{description}

So, $h	=	\left(d_{m} + \lfloor \frac{13(m + 1)}{4} \rfloor + K + \lfloor \frac{K}{4} \rfloor +  + \lfloor \frac{J}{4} \rfloor - 2J\right) mod 7$.

\subsection{Temperature}
\begin{description}
	\item[$\deg C$ to $\deg F$]	Multiply by 9, divide by 5 and add 32.
	\item[$\deg F$ to $\deg C$]	Deduct 32, multiply by 5 and divide by 9.
\end{description}
Alternatively, we can multiply \textit{(divide)} by 1.8 (i.e.,  9/5) to convert from $\deg C$ to $\deg F$ \textit{(or vice-versa)}.

\end{multicols*}
%% -----------------------------
%% Fin du document
%% -----------------------------
\end{document}
