\documentclass[10pt, french]{article}
%% -----------------------------
%% Préambule
%% -----------------------------
% Extra note : this preamble creates document that are meant to be used inside the multicols environment. See the documentation on internet for further information.

%% -----------------------------
%% Encoding packages
%% -----------------------------
\usepackage[utf8]{inputenc}
\usepackage[T1]{fontenc}
\usepackage{babel}
\usepackage{lmodern}
%
%%% -----------------------------
%%% Variable definition
%%% -----------------------------
%\def\auteur{Alec James van Rassel}
%\def\BackgroundColor{white}
%
%%% -----------------------------
%%% Margin and layout
%%% -----------------------------
%% Determine the margin for cheatsheet
\usepackage[hmargin=1cm, vmargin=1.7cm]{geometry}
\usepackage{multicol}

%% -----------------------------
%% URL and links
%% -----------------------------
\usepackage{hyperref}
\hypersetup{colorlinks = true, urlcolor = white, linkcolor = black}

%% -----------------------------
%% Document policy (uncomment only one)
%% -----------------------------
%	\usepackage{concrete}
%	\usepackage{mathpazo}
%	\usepackage{frcursive} %% permet d'écrire en lettres attachées
%	\usepackage{aeguill}
%	\usepackage{mathptmx}
%	\usepackage{fourier} 

%% -----------------------------
%% Math configuration
%% -----------------------------
\usepackage[fleqn]{amsmath}
\usepackage{amsthm,amssymb,latexsym,amsfonts}
\usepackage{empheq}
\usepackage{numprint}
\usepackage{dsfont} % Pour avoir le symbole du domaine Z

% Mathematics shortcuts

\newcommand{\reels}{\mathbb{R}}
\newcommand{\entiers}{\mathbb{Z}}
\newcommand{\naturels}{\mathbb{N}}
\newcommand{\eval}{\biggr \rvert}
\usepackage{cancel}
\newcommand{\derivee}[1]{\frac{\partial}{\partial #1}}
\newcommand{\prob}[1]{\Pr \left( #1 \right)}
\newcommand{\esp}[1]{\mathrm{E} \left[ #1 \right]} % espérance
\newcommand{\variance}[1]{\mathrm{Var} \left( #1   \right)}
\newcommand{\covar}[1]{\mathrm{Cov} \left( #1   \right)}
\newcommand{\laplace}{\mathcal{L}}
\newcommand{\deriv}[2][]{\frac{\partial^{#1}}{\partial #2^{#1}}}
\newcommand{\e}[1]{\mathrm{e}^{#1}}
\newcommand{\te}[1]{\text{exp}\left\{#1\right\}}
\DeclareMathSymbol{\shortminus}{\mathbin}{AMSa}{"39}

% To indicate equation number on a specific line in align environment
\newcommand\numberthis{\addtocounter{equation}{1}\tag{\theequation}}

%
% Actuarial notation packages
%
\usepackage{actuarialsymbol}
\usepackage{actuarialangle}

%
% Matrix notation for math symbols (\bm{•})
%
\usepackage{bm}
% Matrix notation variable (bold style)
\newcommand{\matr}[1]{\mathbf{#1}}


%% -----------------------------
%% tcolorbox configuration
%% -----------------------------
\usepackage[most]{tcolorbox}
\tcbuselibrary{xparse}
\tcbuselibrary{breakable}
%%	-----------------------------
%%
%%	Arguments
%%	+	breakable: allows box to be split over several pages
%%	+	segmentation style: To customize the \tcbline seperator
%%	
%%	-----------------------------
%%
%%
%% Coloured box "definition" for definitions
%%
%%
%% Coloured box "definition2" for definitions
%%
\DeclareTColorBox{definitionNOHFILL}{ o }				% #1 parameter
{
	colframe=blue!60!green,colback=blue!5!white, % color of the box
	pad at break* = 0mm, 						% to split the box
	title = {#1},
	before title = {\faBook \quad },
	breakable
}
%%
%% Coloured box "definition2" for definitions
%%
\DeclareTColorBox{definitionNOHFILLsub}{ o }				% #1 parameter
{
	colframe=blue!40!green,colback=blue!5!white, % color of the box
	pad at break* = 0mm, 						% to split the box
	title = {#1},
	before title = {\faNavicon \quad }, %faBars  faGetPocket
	breakable
}
\DeclareTColorBox{theorems}{ o}			% #1 parameter
{
	enhanced,
	title = #1,
	colback=bluebell, % color of the box
	colframe=blue(pigment),
	colbacktitle=blue!80!black,
	fonttitle = \bfseries,
	boxed title style={size=small,colframe=purple!50!black} ,
	attach boxed title to top center = {yshift=-3mm,yshifttext=-1mm},
	left=0pt,
  	right=0pt,
    box align=center,
    ams align*
%  	top=-10pt
}
\DeclareTColorBox{distributions}{ o }			% #1 parameter
{
	enhanced,
	title = #1,
	colback=ashgrey, % color of the box
%	colframe=blue(pigment),
%	colframe=lightgray,	
	colbacktitle=aurometalsaurus,
	fonttitle = \bfseries,
	boxed title style={size=small,colframe=arsenic} ,
	attach boxed title to top center = {yshift=-3mm,yshifttext=-1mm},
%	left=0pt,
%  	right=0pt,
%    box align=center,
%    ams align*
%  	top=-10pt
}
\DeclareTColorBox{astuces}{ o }			% #1 parameter
{
	enhanced,
	title = #1,
	colback=beaublue, % color of the box
%	colframe=blue(pigment),
	colframe=ballblue,	
	colbacktitle=aurometalsaurus,
	fonttitle = \bfseries,
	boxed title style={size=small,colframe=arsenic} ,
	attach boxed title to top center = {yshift=-3mm,yshifttext=-1mm},
%	left=0pt,
%  	right=0pt,
%    box align=center,
%    ams align*
%  	top=-10pt
}
\DeclareTColorBox{outcomes}{ o }			% #1 parameter
{
	enhanced,
	title = #1,
	colback=bluebell, % color of the box
%	colframe=blue(pigment),
%	colframe=asparagus,	
	colbacktitle=airforceblue,
	fonttitle = \bfseries,
	boxed title style={size=small,colframe=arsenic} ,
	attach boxed title to top center = {yshift=-3mm,yshifttext=-1mm},
%	left=0pt,
%  	right=0pt,
%    box align=center,
%    ams align*
%  	top=-10pt
}
\DeclareTColorBox{ASM_chapter}{ o }			% #1 parameter
{
	enhanced,
	title = #1,
	colback=darkseagreen, % color of the box
%	colframe=blue(pigment),
%	colframe=asparagus,	
	colbacktitle=britishracinggreen,
	fonttitle = \bfseries,
	boxed title style={size=small,colframe=arsenic} ,
	attach boxed title to top center = {yshift=-3mm,yshifttext=-1mm},
	segmentation style = {dashed, white},
	breakable
%	left=0pt,
%  	right=0pt,
%    box align=center,
%    ams align*
%  	top=-10pt
}
\DeclareTColorBox{YTB_vids}{ o }			% #1 parameter
{
	enhanced,
	title = #1,
	colback=red_rectangle, % color of the box
%	colframe=blue(pigment),
%	colframe=asparagus,	
	colbacktitle=lava,
	fonttitle = \bfseries,
	boxed title style={size=small,colframe=arsenic} ,
	attach boxed title to top center = {yshift=-3mm,yshifttext=-1mm},
	segmentation style = {dashed, white},
	breakable
%	left=0pt,
%  	right=0pt,
%    box align=center,
%    ams align*
%  	top=-10pt
}
%%
%% Coloured box "algo" for algorithms
%%
\newtcolorbox{algo}[ 1 ]
{
	colback = blue!5!white,
	colframe = blue!75!black,
	fonttitle = \bfseries,title=#1
}
%%
%% Coloured box "formula" for formulas
%%
\newtcolorbox{formula}[ 1 ]
{
	colback = beaublue,
	colframe = airforceblue,
	fonttitle = \bfseries,title=#1
}
%%
%% Coloured box "CHPT_SUMM" pour résumés des chapitres de l'ASM
%%
\newtcolorbox{CHPT_SUMM}[ 1 ]
{
	colback = green!5!white,
	colframe = darkseagreen,
	breakable,
	fonttitle = \bfseries,title=#1
}
%%
%% Coloured box "FORMULA_SUMM" pour résumés des formules de l'ASM
%%
\newtcolorbox{FORMULA_SUMM}[ 1 ]
{
	colback = babyblueeyes,
	colframe = airforceblue,
	breakable,
	fonttitle = \bfseries,title=#1
}
%%
%% Coloured box "YTB_SUMM" pour résumés des vidéos Youtube
%%
\newtcolorbox{YTB_SUMM}[ 1 ]
{
	colback = red!5!white,
	colframe = darkterracotta,
	breakable,
%	    frame hidden,
	fonttitle = \bfseries,title=#1
}
\newtcolorbox[auto counter, list inside = CHPT]{YTB_SUMM_AUTO_NUMB}[ 2 ][]
{
	colback = red!5!white,
	colframe = darkterracotta,
	breakable,
	enhanced,
	fonttitle = \bfseries,
	title = Video~\thetcbcounter: #2, 		
%	title = #2,
	nameref = #2,
	after upper = {\addcontentsline{toc}{subsubsection}{\thetcbcounter: #2}},
%	phantomlabel = {#2},
	#1
}
%%
%% Coloured box "rappel" pour rappel de formules
%%
\DeclareTColorBox{rappel_enhanced}{ o }
{
	enhanced,
	title = #1,
	colback=lightgray, % color of the box
%	colframe=blue(pigment),
%	colframe=arsenic,	
	colbacktitle=arsenic,
	fonttitle = \bfseries,
	breakable,
	boxed title style={size=small,colframe=arsenic} ,
	attach boxed title to top center = {yshift=-3mm,yshifttext=-1mm},
}
%%
%% Coloured box "FORMULA_SUMM" pour résumés des formules de l'ASM
%%
%% -----------------------------
%% Graphics and pictures
%% -----------------------------
\usepackage{graphicx}
\usepackage{pict2e}
\usepackage{tikz}
%%
%%	Creates circle 
%%	Arguments:
%%	+	size
%%	+	colour
%%	
%%	Example:
%%	+	\tikzcircle[green, fill=blue]{1.5pt}
%%	+	\tikzcircle{2pt}
%%
\newcommand{\tikzcircle}[2][red,fill=red]{\tikz[baseline=-0.5ex]\draw[#1,radius=#2] (0,0) circle ;}


%% -----------------------------
%% insert pdf pages into document
%% -----------------------------
\usepackage{pdfpages}

%% -----------------------------
%% Color configuration
%% -----------------------------
\usepackage{color, soulutf8, colortbl}

%
%	Colour definitions
%
\definecolor{ceruleanblue}{rgb}{0.16, 0.32, 0.75}
\definecolor{darkterracotta}{rgb}{0.8, 0.31, 0.36}   % red pastel ish
\definecolor{lava}{rgb}{0.81, 0.06, 0.13}
\definecolor{wildwatermelon}{rgb}{0.99, 0.42, 0.52}  % red ish
\definecolor{bostonuniversityred}{rgb}{0.8, 0.0, 0.0} % rich red
\definecolor{asparagus}{rgb}{0.53, 0.66, 0.42}		% sorta militarygreen but pastel
\definecolor{darkseagreen}{rgb}{0.56, 0.74, 0.56}    % pastel light green
\definecolor{britishracinggreen}{rgb}{0.0, 0.26, 0.15} %dark green
\definecolor{airforceblue}{rgb}{0.36, 0.54, 0.66}	% nice teal blue pastel
\definecolor{babyblueeyes}{rgb}{0.63, 0.79, 0.95}	% pastel blue-ish
\definecolor{applegreen}{rgb}{0.55, 0.71, 0.0}		% green with some aqua
\definecolor{indigo(web)}{rgb}{0.29, 0.0, 0.51}
\definecolor{cobalt}{rgb}{0.0, 0.28, 0.67}
\definecolor{azure(colorwheel)}{rgb}{0.0, 0.5, 1.0}
\definecolor{darkpastelpurple}{rgb}{0.59, 0.44, 0.84}
\definecolor{darkgreen}{rgb}{0.0, 0.2, 0.13}			
\definecolor{burntorange}{rgb}{0.8, 0.33, 0.0}		
\definecolor{burntsienna}{rgb}{0.91, 0.45, 0.32}		
\definecolor{ao(english)}{rgb}{0.0, 0.5, 0.0}		% ACT-2003
\definecolor{amber(sae/ece)}{rgb}{1.0, 0.49, 0.0} 	% ACT-2004
\definecolor{green_rectangle}{RGB}{131, 176, 84}		% ACT-2004
\definecolor{red_rectangle}{RGB}{241,112,113}		% ACT-2004
\definecolor{blue_rectangle}{RGB}{83, 84, 244}		% ACT-2004
\definecolor{blue(pigment)}{rgb}{0.2, 0.2, 0.6}
\definecolor{bluebell}{rgb}{0.64, 0.64, 0.82}
\definecolor{amethyst}{rgb}{0.6, 0.4, 0.8}
\definecolor{amethyst-light}{rgb}{0.6, 0.4, 0.8}
\definecolor{aurometalsaurus}{rgb}{0.43, 0.5, 0.5}
\definecolor{arsenic}{rgb}{0.23, 0.27, 0.29}			%	dark black-grey ish pastel
\definecolor{ashgrey}{rgb}{0.7, 0.75, 0.71}
\definecolor{beaublue}{rgb}{0.74, 0.83, 0.9}
\definecolor{ballblue}{rgb}{0.13, 0.67, 0.8}
\definecolor{lightgray}{rgb}{0.83, 0.83, 0.83}
\definecolor{antiquefuchsia}{rgb}{0.57, 0.36, 0.51}
%
% Useful shortcuts for coloured text
%
\newcommand{\orange}{\textcolor{orange}}
\newcommand{\red}{\textcolor{red}}
\newcommand{\cyan}{\textcolor{cyan}}
\newcommand{\blue}{\textcolor{blue}}
\newcommand{\green}{\textcolor{green}}
\newcommand{\purple}{\textcolor{magenta}}
\newcommand{\yellow}{\textcolor{yellow}}

%% -----------------------------
%% Enumerate environment configuration
%% -----------------------------
%
% Custum enumerate & itemize Package
%
\usepackage{enumitem}
%
% French Setup for itemize function
%
\frenchbsetup{StandardItemLabels=true}
%
% Change default label for itemize
%
\renewcommand{\labelitemi}{\faAngleRight}


%% -----------------------------
%% Tabular column type configuration
%% -----------------------------
\newcolumntype{C}{>{$}c<{$}} % math-mode version of "l" column type
\newcolumntype{L}{>{$}l<{$}} % math-mode version of "l" column type
\newcolumntype{R}{>{$}r<{$}} % math-mode version of "l" column type
\newcolumntype{f}{>{\columncolor{green!20!white}}p{1cm}}
\newcolumntype{g}{>{\columncolor{green!40!white}}m{1.2cm}}
\newcolumntype{a}{>{\columncolor{red!20!white}$}p{2cm}<{$}}	% ACT-2005
% configuration to force a line break within a single cell
\usepackage{makecell}


%% -----------------------------
%% Fontawesome for special symbols
%% -----------------------------
\usepackage{fontawesome}

%
%%% -----------------------------
%%% Footer/Header Customization
%%% -----------------------------
\usepackage{lastpage}
\usepackage{fancyhdr}
\pagestyle{fancy}
%%
%% Page background color
%%
\pagecolor{white}

%% -----------------------------
%% Section Font customization
%% -----------------------------
\usepackage{sectsty}
%\sectionfont{\color{\SectionColor}}
%\subsectionfont{\color{\SubSectionColor}}



\usepackage{ctable}
%% END OF PREAMBLE
% ---------------------------------------------
% ---------------------------------------------
%% -----------------------------
%% Redefine from template
%% -----------------------------
\def\auteur{Alec James van Rassel}
%% -----------------------------
%% Variable definition
%% -----------------------------
\def\cours{Francais}
\def\sigle{alloprof}
%% -----------------------------
%% Colour setup for sections
%% -----------------------------
\def\SectionColor{burntorange}
\def\SubSectionColor{burntsienna}
\def\SubSubSection{burntsienna}
%%%
%%%
%%%
\hypersetup{colorlinks = true, urlcolor = blue, linkcolor = black}
\usepackage{multicol}
\usetikzlibrary{matrix}

\DeclareTColorBox{distributions}{ o }			% #1 parameter
{
	enhanced,
	title = #1,
	colback=ashgrey, % color of the box
%	colframe=blue(pigment),
	breakable,
	colframe=arsenic,	
	colbacktitle=aurometalsaurus,
	fonttitle = \bfseries,
	boxed title style={size=small,colframe=arsenic} ,
	attach boxed title to top center = {yshift=-3mm,yshifttext=-1mm},
%	left=0pt,
%  	right=0pt,
%    box align=center,
%    ams align*
%  	top=-10pt
}
% 
% Débute numérotation des chapitres à 2 pour suivre les notes de Marie-Piere.
% 
\setcounter{section}{1}

%% -----------------------------
%% Début du document
%% -----------------------------
\begin{document}

\begin{multicols*}{2}
\section*{\href{http://www.alloprof.qc.ca/bv/pages/f1128.aspx}{La phrase}}

\subsection*{\href{http://www.alloprof.qc.ca/BV/Pages/f1130.aspx}{Le sujet}}
\begin{description}
	\item[Description technique]	Le sujet est une \textbf{fonction} grammaticale qui régit l'accord du verbe;
	\item[Description moins technique]	C'est le \og groupe \fg{} qui donne au verbe son nom et sa personne;
		\begin{itemize}[leftmargin = *]
		\item	On dit \og groupe \fg{} car la fonction sujet est souvent occupée par un \href{http://www.alloprof.qc.ca/BV/pages/f1235.aspx}{groupe du nom};
		\item	Par exemple \og le petit Louis \fg{}, \og les hommes \fg{}, \og les voitures \fg{}.
		\end{itemize}
	\item[Sur le plan sémantique]	il indique \textbf{de qui} ou \textbf{de quoi} on parle dans la phrase.
\end{description}

\begin{distributions}[La fonction sujet peut être occupée par:]
\begin{description}
%%%	--------------------
%%%	NOTE
%%%	+	De quoi qui me manque en compréhension ici pour GN (AJVR).
%%%	--------------------
	\item[Le groupe nominal]	\textbf{GNs};
	\item[Le pronom]	on dit \textbf{pronom sujet};
		\begin{itemize}[leftmargin = *]
		\item	\textcolor{blue_rectangle}{Elles} veulent se rappeler de ce moment tout leur vie.
		\item	\textcolor{blue_rectangle}{Nous} désirons vous rencontrer dans les plus brefs délais.
		\item	\textcolor{blue_rectangle}{Je} suis confortable dans ce lit.
		\end{itemize}
	\item[Le groupe du verbe à infinitif]	on dit \textbf{groupe infinitif sujet};
		\begin{itemize}[leftmargin = *]
		\item	\textcolor{blue_rectangle}{Manger trois repas par jour} est une habitude de vie saine.
		\item	\textcolor{blue_rectangle}{Se marier} est le rêve de bien des gens.
		\item	\textcolor{blue_rectangle}{Étudier} est la clé à la réussite.
		\item	Les groupes on donc un verbe à l'infinitif comme noyau.
		\end{itemize}
	\item[La subordonnée complétive]	on dit \textbf{subordonnée complétive sujet}
		\begin{itemize}[leftmargin = *]
		\item	\textcolor{blue_rectangle}{Qu'il pleuve} change le programme de ma journée.
		\item	\textcolor{blue_rectangle}{Que tu m'appelles} me comble de joie.
		\item	\textcolor{blue_rectangle}{Que tu sois récompensé} est bien normal.
		\item	Les groupes comportent donc un subordonnant \textit{(qu', que)} et un verbe conjugué \textit{(pleuve, appelles, sois récompensé)}.
		\end{itemize}
\end{description}
\end{distributions}

\begin{astuces}[Trucs pour trouver le sujet dans une phrase]
\begin{enumerate}[leftmargin = *]
	\item	Poser la question \textit{Qu'est-ce qui ?} ou \textit{Qui est-ce qui ?} devant le verbe.
	\item	Encadrer le sujet par \textbf{C'est \dots qui} ou \textbf{Ce sont \dots qui}.
	\item	Remplacer le sujet par un pronom (procédé appelé la \textbf{pronominalisation}).
		\begin{itemize}[leftmargin = *]
		\item	Par exemple, remplacer \og Alec aime apprendre \fg{} par \og il aime apprendre \fg{}.
		\end{itemize}
\end{enumerate}
\end{astuces}

\paragraph{NOTE}	Le sujet de la phrase est \textbf{non-effaçable} sinon elle n'a plus de sens. On  dit donc que le sujet est un \textbf{constituant obligatoire} de la phrase.


\subsection*{\href{http://www.alloprof.qc.ca/BV/Pages/f1131.aspx}{Le prédicat}}

\subsection*{\href{http://www.alloprof.qc.ca/BV/Pages/f1132.aspx}{Le complément de phrase}}



%\begin{tikzpicture}
%\clip node (m) [
%	matrix,
%	matrix of nodes,
%	fill = black!20, % alternating rows color
%	inner sep = 1pt, % width of exterior line
%	nodes in empty cells,
%	nodes = {
%		minimum height = 1cm,
%		minimum width = 2.6cm,
%		anchor = center,
%		outer sep = 0,
%		font = \sffamily
%	},
%	row 1/.style = {
%		nodes = {
%			fill = ballblue,  % header colour
%			text = white,
%			font = \bfseries
%		}
%	},
%%	column 1/.style = {
%%		nodes = {
%%			fill = gray,
%%			text = white,
%%			align = right,
%%			text width = 2.5cm,
%%			text depth = 0.5ex
%%		}
%%	},
%	column 1/.style = {
%		text width = 4cm, 
%		align = center,
%		nodes = {
%			font = \bfseries
%		},
%		every even row/.style = {
%			nodes = {
%				fill = white
%			}
%		}
%	},
%	column 2/.style = {
%		text width = 3cm,
%		align = center,
%		every even row/.style = {nodes = {fill = white}},
%	},
%%	row 1 column 1/.style = {nodes = {fill = gray}},
%	prefix after command = {
%		[rounded corners = 4mm] (m.north east) rectangle (m.south west)
%	}
%] {
%	Norwegian	&	English \\
%	Jeg	&	I	\\
%	du	&	you	\\
%	han	&	he	\\
%	hun	&	she	\\
%	det	&	it	\\
%};
%\end{tikzpicture}
%
%\subsection*{Verbs}
%\begin{tikzpicture}
%\clip node (m) [
%	matrix,
%	matrix of nodes,
%	fill = black!20, % alternating rows color
%	inner sep = 1pt, % width of exterior line
%	nodes in empty cells,
%	nodes = {
%		minimum height = 1cm,
%		minimum width = 2.6cm,
%		anchor = center,
%		outer sep = 0,
%		font = \sffamily
%	},
%	row 1/.style = {
%		nodes = {
%			fill = ballblue,  % header colour
%			text = white,
%			font = \bfseries
%		}
%	},
%%	column 1/.style = {
%%		nodes = {
%%			fill = gray,
%%			text = white,
%%			align = right,
%%			text width = 2.5cm,
%%			text depth = 0.5ex
%%		}
%%	},
%	column 1/.style = {
%		text width = 4cm, 
%		align = center,
%		nodes = {
%			font = \bfseries,
%			fill = lightgray
%		}
%	},
%	column 2/.style = {
%		text width = 3cm,
%		align = center,
%		nodes = {
%			fill = white
%		}
%	},
%	column 3/.style = {
%		text width = 3cm,
%		align = center,
%		nodes = {
%			font = \bfseries,
%			fill = lightgray
%		}
%	},
%	column 4/.style = {
%		text width = 3cm,
%		align = center,
%		nodes = {
%			fill = white
%		}
%	},
%%	row 1 column 1/.style = {nodes = {fill = gray}},
%	prefix after command = {
%		[rounded corners = 4mm] (m.north east) rectangle (m.south west)
%	}
%] {
%	Singular	&	&	Plural	&\\
%	Jeg er	&	I am	&	vi er	&	we are	\\
%	du er	&	you are	&	dere er	&	 you are	\\
%	han, hun, det, er	&	he, she, it, is	&	de er	&	they are	\\
%};
%\end{tikzpicture}

\end{multicols*}
%% -----------------------------
%% Fin du document
%% -----------------------------
\end{document}
